
This is another important layer in our design. This part of the system works as a bridge between the User Interface and the Database of the system. If the data is being received from the UI controller that are coming form different subsections of UI, the Business controller will, depending upon what kind of input it gets, either generate an AR, generate a Qr or does an input validation. The Business Logic will send the data to the data base for more validation and verification or to store the data in database. If Business Logic is getting the instruction to fetch data form Database, then it will contact Database to get particular data, which is then passed on to the UI.

\subsection{Business Controller}
Any kind of data coming and going to Business Logic do so through the Business controller. The data coming to Business Logic will be divided into three different ways, namely AR generation, QR generation or Input Validation related. Depending upon those different input types, Business controller communicates with the Database controller to get particular data or to store that information.

\begin{figure}[h!]
	\centering
 	\includegraphics[width=0.60\textwidth]{images/businesscontroller}
 \caption{Business Controller Subsystem Interfaces}
\end{figure}

\subsubsection{Assumptions}

\begin{itemize}
    
    \item UI is working fine and UI controller has well established connection with the Business Logic.
    \item Database is working fine and DB controller has well established connection with the Business Logic.
\end{itemize}

\subsubsection{Responsibilities}
Following are the responsibilities of the Business Logic:
\begin{itemize}
    \item Creating QR for any new item to be stored in the inventory.
    \item Creating AR for any new item to be stored in the inventory.
    \item Validate the input coming from the UI.
    \item Send QR generated to be stored in Database.
    \item Send AR generated  to be stored in the Database.
    \item Fetch required data from Database that are required by the UI.
\end{itemize}


\subsubsection{Business Controller Subsystem Interfaces}
Each of the inputs and outputs for the subsystem are defined here. 

\begin {table}[H]

\begin{center}
    \begin{tabular}{ | p{1cm} | p{6cm} | p{3cm} | p{3cm} |}
    \hline
    ID & Description & Inputs & Outputs \\ \hline
     \#11 & UI\_Controller/Business\_Controller & \pbox{3cm}{User input \\ label} & \pbox{3cm}{N/A}  \\ \hline
    \#12 & Business\_Controller/UI\_Controller & \pbox{3cm}{N/A} & \pbox{3cm}{Success Message \\ Failure Message \\ Item image}  \\ \hline
    \#13 & Business Controller/AR Generator & \pbox{3cm}{Item from database} & \pbox{3cm}{\\}  \\ \hline
    \#14 & AR Generator/Business Controller & \pbox{3cm}{N/A} & \pbox{3cm}{Item Image}  \\ \hline
    \#15 & Business Controller/QR Generator & \pbox{3cm}{Item Description form UI\_Controller} & \pbox{3cm}{N/A}  \\ \hline
    \#16 & QR Generator/Business Controller & \pbox{3cm}{N/A} & \pbox{3cm}{Item description \\ QR code}  \\ \hline
    \#17 & Business Controller/UI Validation & \pbox{3cm}{Any input from UI\_Controller} & \pbox{3cm}{N/A}  \\ \hline
    \#18 & UI Validation/Business Controller & \pbox{3cm}{N/A} & \pbox{3cm}{Valid or Invalid }  \\ \hline
    \#19 & Business Controller/DB\_Controller & \pbox{3cm}{Any input from UI\_Controller if valid} & \pbox{3cm}{N/A}  \\ \hline
    \#20 & DB\_Controller/Business Controller & \pbox{3cm}{N/A} & \pbox{3cm}{Item\\ Success or failure Message}  \\ \hline
    \end{tabular}
    \caption {Business Controller Subsystem Interfaces} 
\end{center}
\end{table}

\subsection{AR Generator}
AR generator is the part of Business logic that generates AR when a customer wants to store an item to the shelf. The generated AR is then used when the customer wants to search the item. When the customer wants to search an item, he will scan bar-code or enter manually. Then the program will display the AR right on the bar-code of the item. 

\begin{figure}[h!]
	\centering
 	\includegraphics[width=0.60\textwidth]{images/argenerator}
 \caption{AR generator description diagram}
\end{figure}

\subsubsection{Assumptions}
\begin{itemize}
    \item item is in the database
    \item User has a device with working camera.
    \item User has stored valid item image for the item.
\end{itemize}

\subsubsection{Responsibilities}
AR generator produces the AR and when a customer scans a bar-code for the item he is searching, the program will display the AR on the Bar-code.

\subsubsection{AR Generator Subsystem Interfaces}


\begin {table}[H]

\begin{center}
    \begin{tabular}{ | p{1cm} | p{6cm} | p{3cm} | p{3cm} |}
    \hline
    ID & Description & Inputs & Outputs \\ \hline
    \#13 & Business Controller/AR Generator & \pbox{3cm}{Item from database} & \pbox{3cm}{N/A}  \\ \hline
    \#14 & AR Generator/Business Controller & \pbox{3cm}{N/A} & \pbox{3cm}{Item Image}  \\ \hline
    \end{tabular}
    \caption {AR Generator Subsystem Interfaces} 
\end{center}
\end{table}

\subsection{QR Generator}
Based on description of the item customer provides, QR generator generate QR.This QR is then store in the database along with the  description of the item.

\begin{figure}[h!]
	\centering
 	\includegraphics[width=0.60\textwidth]{images/qrgenerator}
 \caption{QR generator description diagram}
\end{figure}

\subsubsection{Assumptions}
\begin{itemize}
    \item Customer inputs the right description of the item to  that is to be stored in inventory.
    \item Item should be new to the inventory.
    
\end{itemize}

\subsubsection{Responsibilities}
QR generator is responsible of generating new QR for each new item that is being added to the inventory. This shall help in giving new identity to the item that is to be stored as well as make it easy to access the item stored in the inventory.

\subsubsection{QR Generator Subsystem Interfaces}

\begin {table}[H]

\begin{center}
    \begin{tabular}{ | p{1cm} | p{6cm} | p{3cm} | p{3cm} |}
    \hline
    ID & Description & Inputs & Outputs \\ \hline
    \#15 & Business Controller/QR Generator & \pbox{3cm}{Item Description form UI\_Controller} & \pbox{3cm}{N/A}  \\ \hline
    \#16 & QR Generator/Business Controller & \pbox{3cm}{N/A} & \pbox{3cm}{Item description \\ QR code}  \\ \hline
    \end{tabular}
    \caption {QR Generator Subsystem Interfaces} 
\end{center}
\end{table}

\subsection{UI Input Validation}
This section will validate format and types of input from user. It will reject the inputs if user's input is wrong format and suggest the correct format of the input.Provide guidance on how to fix any errors, don't just tell users what they did wrong. It will prevent from unauthorized SQL injection into the data base.
\begin{figure}[h!]
	\centering
 	\includegraphics[width=0.60\textwidth]{images/uivalidation}
 \caption{UI Input Validation description diagram}
\end{figure}

\subsubsection{Assumptions}
\begin{itemize}
    \item User can input item information.
    \item If  team’s assumption was correct, move forward to the next assumption.
\end{itemize}

\subsubsection{Responsibilities}
\begin{itemize}
   
    \item Conduct an experiment like the one listed to see if they were correct.
    \item Reject the SQL injection 
   
\end{itemize}


\subsubsection{UI Input Validation Subsystem Interfaces}
 Inputs and outputs for the UI validation are defined here. 
\begin {table}[H]

\begin{center}
    \begin{tabular}{ | p{1cm} | p{6cm} | p{3cm} | p{3cm} |}
    \hline
    ID & Description & Inputs & Outputs \\ \hline
    \#17 & Business Controller/UI Validation & \pbox{3cm}{Any input from UI\_Controller} & \pbox{3cm}{N/A}  \\ \hline
    \#18 & UI Validation/Business Controller & \pbox{3cm}{N/A} & \pbox{3cm}{Valid or Invalid }  \\ \hline
    \end{tabular}
    \caption {UI Input Validation Subsystem Interfaces} 
\end{center}
\end{table}

